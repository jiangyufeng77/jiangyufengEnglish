\documentclass[11pt,oneside,a4paper]{article}
\usepackage{indentfirst,graphicx}
\setlength{\parindent}{1em}
\title{Image Sensing and Acquisition}
\author{Yufeng Jiang}
\date{\today}
\begin{document}
\maketitle
Most of the images are generated by the combination of an illumination source and the reflection of energy from source. Figure 1 shows the three principal sensor arrangements used to transform illumination energy into digital images, including single imaging sensor, line sensor and array sensor. \\
\includegraphics[width=20em]{1.jpeg}\\
Figure 1\footnote{single imaging sensor, line sensor and array sensor.}\\
\\
\begin{tabular}{|c|c|c|}
  \hline
  \multicolumn{3}{|c|}{Image Sensing}\\ \hline
  single imaging sensor & line sensor & array sensor \\ \hline
\end{tabular} \\
\\
\indent The most familiar sensor of a single sensor is the photodiode, which is constructed of silicon materials. Its output voltage is increasing or decreasing as the light increases or decreases. The character of this type is inexpensive and high-precision but slow. \\
\indent The in-line arrangement of sensors in the form of a sensor strip is used much more frequently than the single sensor. The strip provides imaging elements in one direction. This type of sensors can be used in the aircraft, medical and industrial imaging to obtain cross-sectional images. \\
\indent The third type is individual sensors arranged in the form of a 2-D array. The most broadest application is used in digital cameras because numerous electromagnetic and some ultrasonic sensing devices frequently are arranged in an array format.\\
\end{document}
