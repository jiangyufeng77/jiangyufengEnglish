\documentclass[a4paper,11pt]{article}
\usepackage{graphicx,float,indentfirst}
\usepackage{amsmath}
\setlength{\parindent}{1em}
\setlength{\columnsep}{5pt}
\setlength{\columnseprule}{0.4pt}
\title{An Introduction to the Mathematical Tools Used in Digital Image Processing}
\author{Yufeng Jiang}
\date{\today}
\begin{document}
\maketitle
In this article, we will be introduced the various mathematical tools about digital image processing, and be helped to use them to a variety of basic image-processing tasks.\\
\section{Array versus Matrix Operations}
\indent Images can be viewed equivalently as matrices. In fact, there are many situations
in which operations between images are carried out using matrix theory. Thus, it is the reason why a clear distinction must be made between array and matrix operation. For example, consider the following 2 * 2 images:\\
$$
\left[
\begin{matrix}
  a_{11} & a_{12} \\
  a_{21} & a_{22} \\
  \end{matrix} 
  \right]
  \left[
  \begin{matrix}
  b_{11} & b_{12} \\
  b_{21} & b_{22} \\
  \end{matrix}
  \right]
$$
\indent The array product of these two images is
$$
\left[ 
\begin{matrix}
  a_{11} & a_{12} \\
  a_{21} & a_{22} \\
  \end{matrix}
  \right]  
  \left[ 
  \begin{matrix}
  b_{11} & b_{12} \\
  b_{21} & b_{22} \\
  \end{matrix}
  \right]  =
  \left[
  \begin{matrix}
  a_{11}b_{11} & a_{12}b_{12} \\
  a_{21}b_{21} & a_{22}b_{22} \\
  \end{matrix}
    \right]
 $$
\section{Linear versus Nonlinear Operations}
\indent One of the most important classifications of an image-processing method is whether it is linear or nonlinear. Consider a general operator, H, that produces an output image, and g(x, y), for given input image, f(x, y):\\
\begin{gather}
H[f(x, y)] = g(x, y)
\end{gather}
\indent H is said to be a linear operator if\\
\begin{equation}
\begin{split}
H[a_{i}f_{i}(x, y) + a_{j}f_{j}(x, y)] &= a_iH[f_i(x, y)+a_jH[f_j(x, y)]] \\
                                       &=a_ig_i(x, y)+a_jg_j(x, y)
\end{split}
\end{equation}
\indent Linear operations are exceptionally important because they are based on a large number of theoretical and practical results that are applicable to image processing\\
\section{Arithmetic Operations}
\begin{figure}[htbp]
\centering
\begin{minipage}[t]{0.48\textwidth}
\centering
\includegraphics[width=6cm]{1.jpeg}
\caption{Mask image.}
\label{fig:1}
\end{minipage}
\begin{minipage}[t]{0.48\textwidth}
\centering
\includegraphics[width=6cm]{2.jpeg}
\caption{A live image.}
\label{fig:2}
\end{minipage}
\end{figure}
\begin{figure}[htbp]
\centering
\begin{minipage}[t]{0.48\textwidth}
\centering
\includegraphics[width=6cm]{3.jpeg}
\caption{Difference between 1 and 2.}
\label{fig:3}
\end{minipage}
\begin{minipage}[t]{0.48\textwidth}
\centering
\includegraphics[width=6cm]{4.jpeg}
\caption{Enhanced difference image.}
\label{fig:4}
\end{minipage}
\end{figure}
\indent Arithmetic operations beetween images are array operations that arithmetic operations are carried out between corresponding pixel pairs. Figure \ref{fig:1} shows a mask X-ray image of the top of a patient's head prior to injection of an iodine medium into the bloodstream. Figure \ref{fig:2} is a sample of a live image taken after the medium was injected. Figure \ref{fig:3} is the difference between \ref{fig:1} and \ref{fig:2}. And Figure \ref{fig:4} is a clear map of how the medium is propagating through the blood vessels in the subject's brain.\\
\end{document}




