\documentclass[11pt,oneside,a4paper]{article}
\usepackage{indentfirst}
\setlength{\parindent}{1em}
\title{Components of an Image Processing System}
\author{Yufeng Jiang}
\date{\today}
\begin{document}
\maketitle
Although large-scale image processing systems are still being sold, the trend continues toward miniaturizing smaller computers with specialized image processing hardware. Figure 1 shows the basic components that comprising a typical general-purpose system used for digital image processing.\\
\\
\begin{tabular}{|c|c|c|}
  \hline
  \multicolumn{3}{|c|}{Network}  \\ \hline
  Image displays & Computers & Mass storage \\ \hline
  Hardcopy & Specialized image processing hardware & Image processing software \\ \hline
  \multicolumn{3}{|c|}{Imagae sensors} \\ \hline
\end{tabular}
  \\
  \\
  \indent With reference to sensing, there are two elements are required to acquire digital images. The one is a physical device that is sensitive to energy that is radiated by images. And the other is digitizer that can convert outputs into digital forms. \\
\indent Specialized image processing hardware usually consists of the digitizer and hardware that performs other primitive operations. The computer is a general-purpose computer and can range from a PC to a supercomputer. Software for image processing consists of specialized modules that perform specific tasks. These modules are integrated by sophisticated software. Mass storage capability is necessary in image processing applications, and it can be divided into three principle categories including short-term storage, on-line storage and archival storage. \\
\indent Image displays are mainly color TV monitors that are driven by the outputs of image and graphics display cards. The purpose of hardcopy devices is recording images including laser printers, film cameras, heat-sensitive devices, inkjet units and digital units. Networking is almost a default function in any computer system.
\end{document}