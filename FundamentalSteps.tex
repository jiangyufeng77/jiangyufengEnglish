\documentclass[11pt,oneside]{article}
\usepackage{indentfirst}
\setlength{\parindent}{1em}
\title{Fundamental Steps in Digital Image Processing}
\author{Yufeng Jiang}
\date{\today}
\begin{document}
\maketitle
This article tells two broad categories, one is method whose input and output are images, and the other is method whose input may be image but output is attribute extracted from the image.\\
\indent The first process is image acquisition which could be as simple as being given an image that is already in digital form. Generally, the image acquisition stage involves preprocessing, such as scaling. The second step is image enhancement which purpose is more suitable than the original for a specific application. In this part, the most important is specific, because it establishes the image enhancement that is orient to problems firstly. Thus, enhancement techniques are so varied and use so many different image processing approaches that it is difficult to define a general method using on it. Image restoration is an area that also deals with improving the appearance of an image. Color image processing has been gaining more importance as the increasing use of digital image of the Internet, and color is used as the basis for extracting features of a image. Compression deals with to reduce the storage required to save an image, or the bandwidth required to transmit it. Morphological processing is watershed, because it deals with to extract image components from images that can output image attributes. \\
\indent Segmentation mainly partition an image into constituent parts. In general, autonomous segmentation is one of the most difficult tasks in digital image processing. When we work on segmentation, the more accurate the segmentation, the more likely recognition is to succeed. In representation and description, the first decision is whether the data should be represented as a boundary or a complete region. Recognition is the process that assigns a label to an object based on its descriptions.
\end{document}