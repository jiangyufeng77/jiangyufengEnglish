\documentclass[11pt,oneside]{article}
\usepackage{indentfirst}
\setlength{\parindent}{1em}
\title{Examples fo Fields that Use Digital Image Processing}
\author{Yufeng Jiang}
\date{\today}
\begin{document}
\maketitle
Today, there is almost no area of technical endeavor that is not impacted in some way by digital image processing. The areas of application of digital image processing are varied. \\
\indent Categorizing images is one of the simplest ways to develop a basic understanding of the extent of the digital image processing applications. The principle energy source is the electromagnetic energy spectrum, and other sources are acoustic, ultrasonic, and electronic. Here are two examples of fields that use digital image processing.\\
\indent The major uses of imaging based on gamma rays are nuclear medicine and astronomical observations. In nuclear medicine, we mainly use it through injecting a patient with a radioactive isotope to locate sites of bone pathology. Similarly, the best known use of X-rays is medical diagnostics, but they are also used in industry and other areas through using an X-ray tube, which is a vacuum tube with a cathode and anode. Angiography is another major application, and it is used to obtain images of blood vessels through inserting a catheter to see any irregularities or blockages. The third important use of X-rays in medical imaging is computerized axial tomography (CAT). Each CAT image is a slice that taken perpendicularly through the patient, and lots of slices are generated as the patient is moved in a longitudinal direction.
\end{document}