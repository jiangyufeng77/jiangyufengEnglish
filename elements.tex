\documentclass[11pt,oneside]{article}
\usepackage{indentfirst,graphicx}
\setlength{\parindent}{1em}
\title{Elements of Visual Perception}
\author{Yufeng Jiang}
\date{\today}
\begin{document}
\maketitle
Although the basic of digital image processing is mathematical and probabilistic formulations, human intuition and analysis play a central role to choose one technique against others. This choice depends on subjective visual judgments.\\
\\
\includegraphics[width=20em]{/Users/jiangyufeng/Desktop/humanEye.jpeg}\\
Figure 1: Simplified diagram of a cross section of the human eye.\\
\\
\indent Figure 1 shows a simplified horizontal cross section of the human eye. The eye is nearly a sphere and consists of three membranes including the cornea and sclera, the choroid and the retina enclose the eye. The choroid contains a network of blood vessels as a major source of nutrition to the eye. The lens is made up of concentric layers of fibrous cells, and it is sensitive to the light. The retina lies in the innermost of eyes. It is mainly foundation is image the light which radiate from an object outside the eye when the eye is properly focused. There are two classes of receptors under the retina including cones and rods. The cones are highly sensitive to color, and the rods that are sensitive to low levels of illumination serve to give a general, overall picture of the field of view. The future discussions believe that the basic ability of the eye to resolve detail certainly is comparable to current electronic imaging sensors.\\
\\
\includegraphics[width=20em]{/Users/jiangyufeng/Desktop/2.jpeg}\\
Figure 2: Graphical representation of the eye looking at a plam tree.\\
\\
\indent When it comes to the image formation of the eye, we need to think the image formation of the camera. In an ordinary camera, the lens has a fixed focal length and various distances between the lens and the imaging plane. Similarly, this foundation is achieved by the fibers in the ciliary body. The figure 2 illustrates how to obtain the dimensions of an image formed on the retina. 
\end{document}