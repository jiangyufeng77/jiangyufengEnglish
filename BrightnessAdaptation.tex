\documentclass[11pt,oneside]{article}
\usepackage{indentfirst,graphicx}
\setlength{\parindent}{1em}
\title{Brightness Adaptation and Discrimination}
\author{Yufeng Jiang}
\date{\today}
\begin{document}
\maketitle
The ability of eyes to discriminate between different intensity levels plays an important role in presenting image processing results because digital images are displayed as a set of intensities. The range of light intensity that human visual system can adapt to is numerous.\\
\\
\includegraphics[width=20em]{/Users/jiangyufeng/Desktop/1.jpeg}\\
Figure 1\footnote{Range of subjectibe brightness sensations showing a particular adaptation level.} \\
\indent The figure 1 depicts the relationship between the light intensity and subjective brightness. Because the visual system cannot operate over such a range simultaneously, it changes its overall sensitivity. This phenomenon calls brightness adaptation. Eyes can {}discriminate the total range of distinct intensity levels is smaller than the total adaptation range. \\
\includegraphics[width=20em]{/Users/jiangyufeng/Desktop/2.jpeg}\\
Figure 2\footnote{Typical Weber ratio as a function of intensity.}\\
\indent The ability of the eye to discriminate between changes in light intensity at any specific adaptation level is important, too. The curve of the figure 2 indicates that brightness discrimination is bad at low levels of illumination, and it improves significantly as background illumination increases. \\
\indent If the background illumination holds constant and the intensity of the other source is allowed to vary from never being perceived to always being perceived, the observer can discriminate the difference twenty varies of intensity. Thus, we can conclude that the eye is capable to discriminate among a broader overall intensity.

\end{document}



