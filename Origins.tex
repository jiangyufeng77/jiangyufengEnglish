\documentclass[11pt,oneside]{article}
\usepackage{indentfirst}
\setlength{\parindent}{1em}
\title{The Origins of Digital Image Processing}
\author{Yufeng Jiang}
\date{\today}
\begin{document}
\maketitle
One of the first applications of digital images was in newspaper industry. In this term, pictures that were specialized printing equipment coded was for cable transmission, and then reconstructed them at the receiving end. This method was replaced by a technique based on photographic reproduction that was made from tapes perforated at the telegraph receiving terminal.\\
\indent Along with the development of computer, the history of digital images processing was intimately tied to it because digital images requires so much storage and computational power. Some advances including invention of the transistor, the integrated circuit, the IBM of the personal computer, and development of the high-level programming languages and microprocessor led to computers powerful enough to be used for digital image processing. \\
\indent From the 1960s until the present, the field of image processing has grown vigorously. In addition to applications in medicine and the space program, digital image processing techniques now are used in a broad range of applications such as the easier interpretation of X-rays, industry, medicine, and the biological sciences. \\
\indent The continuing decline in the ratio of computer price to performance, the expansion of networking and the Internet have created unprecedented opportunities for continued growth of digital image processing.
\end{document}