\documentclass[11pt,oneside]{article}
\usepackage{indentfirst,graphicx}
\setlength{\parindent}{1em}
\title{Light and the Electromagnetic Spectrum}
\author{Yufeng Jiang}
\date{\today}
\begin{document}
\maketitle
In 1666, Sir Isaac Newton discovered a beam of light consists a continuous spectrum of colors ranging form violet to red. As Figure 1 shows, the color what we perceive in visible light represents a very small portion of the electromagnetic spectrum. On the one end of the spectrum are radio waves, and on the other end of it are gamma rays. The electromagnetic spectrum can be expressed in terms of wavelength, frequency or energy. They are related by the expression: 
$ \lambda=\frac{c}{v} $ where the c is the speed of light. We can conclude from the expression:
$E=hv$ that energy is proportional to frequency.\\
 \\
\includegraphics[width=35em]{/Users/jiangyufeng/Desktop/1.jpeg}\\
Figure 1: The electromagnetic spectrum. The visible spectrum is shown zoomed to facilitate examplanation, but note that the visible spectrum is a rather narrow portion of the EM spectrum.\\
\\
\indent Light is a particular part of electromagnetic radiation that can be sensed by human eyes. In fact, the range of electromagnetic radiation is very broad. It can be divided into six broad regions: violet, blue, green, yellow, orange and red. Every color don not change abruptly, but smoothly. \\
\indent Light that is void of color is called monochromatic light. The term gray level is used to denote monochromatic intensity, and the range of measured values of monochromatic light from black to white is called gray scale. In contrast, chromatic light is described including radiance, luminance and brightness as three basic quantities.\\
\indent On the short-wavelength end, there are gamma rays and X-rays. As the increase of the wavelength, we can encounter the infrared band. The area which near the infrared band is called the near-infrared region, and the opposite area is called the far-infrared region. The far-infrared region is near by the microwave. In the end, we can get the radio wave.
\begin{thebibliography}{9}
\bibitem{pa} Rafael C.~Gonzalez and Richard E.~Woods, ``Digital Image Processing,'' Publishing House of Electronics Industry (2013)
\end{thebibliography}
\end{document}